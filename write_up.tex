\documentclass{article}
\title{Applications of Graph Theory and Combinatorics in Computer Science}
\author{Dylan Galea}

\usepackage{cite}
\usepackage{amsmath}
\usepackage{amsthm}

\newtheorem{theorem}{Theorem}[subsection]
\newtheorem{definition}{Definition}[subsection]
\newtheorem{lemma}[theorem]{Lemma}

\begin{document}
\tableofcontents
\maketitle
\section{The Travelling Salesman Problem}
\subsection{Important Definitions and Problem Background}
To define the Travelling Salesman Problem , first the concept of a Hamiltonian Cycle must be defined. This concept is defined in definition \ref{Hamiltonian Cycle} below.
\begin{definition}[Hamiltonian Cycle]
\label{Hamiltonian Cycle}
Given a graph G(V,E), a Hamiltonian Cycle in G is a cycle in G such that $\forall$ v $\in$ V, v is in the cycle and is visited only once . A graph that contains a Hamiltonian cycle is called a Hamiltonian Graph \cite{weisstein_2018}.
\end{definition}
The Travelling Salesman problem can now be defined as shown in definition \ref{Travelling Salesman Problem} below.
\begin{definition}[Travelling Salesman Problem]
\label{Travelling Salesman Problem}
Given a simple graph G(V,E) such that $\forall$ v,w $\in$ V ,v $\ne$ w $\{v,w\}$ $\in$ E, the Travelling Salesman Problem is the task of finding a minimum weight Hamiltonian Cycle in G \cite{geeksforgeeks_2018}.
\end{definition}
Definition \ref{Travelling Salesman Problem} suggests that in the Travelling Salesman Problem it is already known that the graph to be evaluated is Hamiltonian, otherwise a minimum weight Hamiltonian Cycle can never be found.This uncertainty can be tackled by proving
that any complete graph on more than 3 vertices is Hamiltonian. This fact is proved in lemma \ref{Kn is Hamiltonian} below.
\begin{lemma}
\label{Kn is Hamiltonian}
For n $>$= 3, The complete graph on n vertices is Hamiltonian
\end{lemma}
\begin{proof}
Let Kn be the complete graph on n $>=$ 3 vertices labelled v1,v2,...,vn. Order all the vertices in the order v1,v2,...,vn with no repetitions of vertices.Then C=(v1 v2 ... vn v1) must be a cycle in Kn because because $\forall$ vi,vj $\in$ C, vi $\ne$ vj then $\{vi,vj\}$ $\in$ E(Kn). Also since $\forall$ v $\in$ V(Kn) ,v is a vertex in the cycle with only one occurence in C(except for v1 which has 2 occurences) then C must be a Hamiltonian Cycle in Kn . Thus Kn must be Hamiltonian.
\end{proof}




ToDo define NP HARD problems .. link with hamiltonian cycle and do example
\newpage{}
\bibliography{bibliography}
\bibliographystyle{IEEEtran}
\end{document}