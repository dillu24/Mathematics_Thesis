\documentclass{article}
\title{Applications of Graph Theory and Combinatorics in Computer Science}
\author{Dylan Galea}

\usepackage{cite}
\usepackage{amsmath}

\newtheorem{definition}{Definition}[section]

\begin{document}
\maketitle
\section{The Travelling Salesman Problem}
To define the Travelling Salesman Problem , first the concept of a Hamiltonian Cycle must be defined.
\begin{definition}[Hamiltonian Cycle]
Given a graph G(V,E), a Hamiltonian Cycle in G is a cycle in G such that $\forall$ v $\in$ V, v is in the cycle and is visited only once . A graph that contains a Hamiltonian cycle is called a Hamiltonian Graph \cite{weisstein_2018}.
\end{definition}
The Travelling Salesmann Problem can now be defined as follows.
\begin{definition}[Travelling Salesman Problem]
Given simple graph G(V,E) were V is the list of vertices(cities) in G and E is the set of weighted edges between every 2 vertices of G , the problem is to find a minimum weight Hamiltonian Cycle in G \cite{geeksforgeeks_2018}.
\end{definition}
ToDo define NP HARD problems .. link with hamiltonian cycle and do example
\newpage{}
\bibliography{bibliography}
\bibliographystyle{IEEEtran}
\end{document}