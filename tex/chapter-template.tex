%-----------------------------------------------------------------------
% Beginning of chapter-template.tex
%-----------------------------------------------------------------------
%
%    This is a template file for monographs prepared with AMS author
%    packages, for use with AMS-LaTeX.  Separate chapters should be
%    included at the appropriate position.
%
%    Templates for various common text, math and figure elements are
%    given following the \end{document} line.
%
%    Start by copying this file to <filename>.tex, using a distinctive
%    name suitable for your book in place of <filename>.  This will
%    be the driver file for your book.
%
%%%%%%%%%%%%%%%%%%%%%%%%%%%%%%%%%%%%%%%%%%%%%%%%%%%%%%%%%%%%%%%%%%%%%%%%

%    Replace amsbook by the documentclass code for the monograph series.
\documentclass{amsbook}

%    If you need symbols beyond the basic set, uncomment this command.
\usepackage{amssymb}

%    If your book includes graphics, or such features as rotation or
%    scaling, uncomment this command.
\usepackage{graphicx}

%    If the book includes commutative diagrams:
%\usepackage[cmtip,all]{xy}

%    If you are using the author-year citation style:
%\usepackage{natbib}

%    Include other referenced packages here.
\usepackage{amsmath}

%    For use when working on individual chapters
%\includeonly{}

%    As set up here, all theorem-class objects will be numbered with
%    the same counter, starting with 1 at every new chapter; numbers
%    will have the form <chapter>.<theorem>.  This may be changed if
%    the author prefers.
\newtheorem{theorem}{Theorem}[chapter]
\newtheorem{lemma}[theorem]{Lemma}

\theoremstyle{definition}
\newtheorem{definition}[theorem]{Definition}
\newtheorem{example}[theorem]{Example}
\newtheorem{xca}[theorem]{Exercise}

\theoremstyle{remark}
\newtheorem{remark}[theorem]{Remark}

\numberwithin{section}{chapter}
\numberwithin{equation}{chapter}

%    For a single index; for multiple indexes, see the manual
%    "AMS Author Handbook, Monograph Classes", included in the
%    author package).
%    Do not \usepackage{makeidx}; all facilities are contained
%    within the AMS document classes.
\makeindex

\begin{document}

\frontmatter

\title{boq}

%    Remove any unused author tags.

%    author one information
\author{boq}
\address{boq}
\curraddr{boq}
\email{boq}
\thanks{boq}

%    author two information
%\author{}
%\address{}
%\curraddr{}
%\email{}
%\thanks{}

%    If any version of the Mathematics Subject Classification other
%    than the 2010 edition appears, then you have an old version
%    of the AMS-LaTeX collection and need to upgrade.  Download from
%    http://www.ams.org/tex/amslatex.html .
\subjclass[2010]{Primary }

\keywords{iva}

\date{iva}

\maketitle

%    Dedication.  If the dedication is longer than a line or two,
%    remove the centering instructions and the line break.
%\cleardoublepage
%\thispagestyle{empty}
%    If this book uses the documentclass stml-l or mmono-s, change
%    13.5pc to 10.5pc.
%\vspace*{13.5pc}
%\begin{center}
%  Dedication text (use \\[2pt] for line break if necessary)
%\end{center}
%\cleardoublepage

%    Change page number to 7 if a dedication is present.
\setcounter{page}{5}

\tableofcontents

%    Include unnumbered chapters (preface, acknowledgments, etc.) here.
%\include{}

\mainmatter
%    Include main chapters here.
%-----------------------------------------------------------------------
% Beginning of chapter.tex
%-----------------------------------------------------------------------
%
%  This is a sample file for use with AMS-LaTeX.  It provides an example
%  of how to set up a file for a book to be typeset with AMS-LaTeX.
%
%  This is the driver file.  Separate chapters should be included at
%  the end of this file.
%
%  ***** DO NOT USE THIS FILE AS A STARTER FOR YOUR BOOK. *****
%  Follow the guidelines in the file chapter.template.
%
%%%%%%%%%%%%%%%%%%%%%%%%%%%%%%%%%%%%%%%%%%%%%%%%%%%%%%%%%%%%%%%%%%%%%%%%

\documentclass{amsbook}

\includeonly{preface,chap1,biblio,index}

\newtheorem{theorem}{Theorem}[chapter]
\newtheorem{lemma}[theorem]{Lemma}

\theoremstyle{definition}
\newtheorem{definition}[theorem]{Definition}
\newtheorem{example}[theorem]{Example}
\newtheorem{xca}[theorem]{Exercise}

\theoremstyle{remark}
\newtheorem{remark}[theorem]{Remark}

\numberwithin{section}{chapter}
\numberwithin{equation}{chapter}

%    Absolute value notation
\newcommand{\abs}[1]{\lvert#1\rvert}

%    Blank box placeholder for figures (to avoid requiring any
%    particular graphics capabilities for printing this document).
\newcommand{\blankbox}[2]{%
  \parbox{\columnwidth}{\centering
%    Set fboxsep to 0 so that the actual size of the box will match the
%    given measurements more closely.
    \setlength{\fboxsep}{0pt}%
    \fbox{\raisebox{0pt}[#2]{\hspace{#1}}}%
  }%
}

\begin{document}
\frontmatter
\title{AMS Monograph Series Sample}

%    Information for first author
\author{Author One}
%    Address of record for the research reported here
\address{Department of Mathematics, Louisiana State University, Baton
Rouge, Louisiana 70803}
%    Current address
\curraddr{Department of Mathematics and Statistics,
Case Western Reserve University, Cleveland, Ohio 43403}
\email{xyz@math.university.edu}
%    \thanks will become a 1st page footnote.
\thanks{The first author was supported in part by NSF Grant \#000000.}

%    Information for second author
\author{Author Two}
\address{Mathematical Research Section, School of Mathematical Sciences,
Australian National University, Canberra ACT 2601, Australia}
\email{two@maths.univ.edu.au}

\date{July 2, 1991}
\subjclass[2010]{Primary 54C40, 14E20;\\Secondary 46E25, 20C20}
\keywords{\texttt{amsbook}, AMS-\LaTeX}

\maketitle

\setcounter{page}{4}
\tableofcontents

%-----------------------------------------------------------------------------
% Beginning of preface.tex
%-----------------------------------------------------------------------------
%
% AMS-LaTeX 1.2 sample file for a monograph, based on amsbook.cls.
% This is a data file input by chapter.tex.
%%%%%%%%%%%%%%%%%%%%%%%%%%%%%%%%%%%%%%%%%%%%%%%%%%%%%%%%%%%%%%%%%%%%%%%%

\chapter*{Preface}

This document is a sample prepared to illustrate the use of the
American Mathematical Society's \LaTeX{} document class
\texttt{amsbook} and publication-specific variants of that class.

This is an example of an unnumbered chapter which can be used for a Preface
or Foreword.

The purpose of this paper is to establish a relationship between an
infinite-dimensional Grassmannian and arbitrary algebraic vector bundles
of any rank defined over an arbitrary complete irreducible algebraic
curve, which generalizes the known connection between the Grassmannian
and line bundles on algebraic curves.

\aufm{Author Name}

%-----------------------------------------------------------------------------
% End of preface.tex
%-----------------------------------------------------------------------------


\mainmatter
%-----------------------------------------------------------------------
% Beginning of chap1.tex
%-----------------------------------------------------------------------
%
%  AMS-LaTeX sample file for a chapter of a monograph, to be used with
%  an AMS monograph document class.  This is a data file input by
%  chapter.tex.
%
%  Use this file as a model for a chapter; DO NOT START BY removing its
%  contents and filling in your own text.
% 
%%%%%%%%%%%%%%%%%%%%%%%%%%%%%%%%%%%%%%%%%%%%%%%%%%%%%%%%%%%%%%%%%%%%%%%%

\part{This is a Part Title Sample}

\chapter{AMS Monograph Series Sample}

\section*{This is an unnumbered first-level section head}
This is an example of an unnumbered first-level heading.

\specialsection*{This is a Special Section Head}
This is an example of a special section head%
%%%%%%%%%%%%%%%%%%%%%%%%%%%%%%%%%%%%%%%%%%%%%%%%%%%%%%%%%%%%%%%%%%%%%%%%
\footnote{Here is an example of a footnote. Notice that this footnote
text is running on so that it can stand as an example of how a footnote
with separate paragraphs should be written.
\par
And here is the beginning of the second paragraph.}%
%%%%%%%%%%%%%%%%%%%%%%%%%%%%%%%%%%%%%%%%%%%%%%%%%%%%%%%%%%%%%%%%%%%%%%%%
.

\section{This is a numbered first-level section head}
This is an example of a numbered first-level heading.

\subsection{This is a numbered second-level section head}
This is an example of a numbered second-level heading.

\subsection*{This is an unnumbered second-level section head}
This is an example of an unnumbered second-level heading.

\subsubsection{This is a numbered third-level section head}
This is an example of a numbered third-level heading.

\subsubsection*{This is an unnumbered third-level section head}
This is an example of an unnumbered third-level heading.

\begin{lemma}
Let $f, g\in  A(X)$ and let $E$, $F$ be cozero sets in $X$.
\begin{enumerate}
\item If $f$ is $E$-regular and $F\subseteq E$, then $f$ is $F$-regular.

\item If $f$ is $E$-regular and $F$-regular, then $f$ is $E\cup
F$-regular.

\item If $f(x)\ge c>0$ for all $x\in E$, then $f$ is $E$-regular.

\end{enumerate}
\end{lemma}

The following is an example of a proof.

\begin{proof} Set $j(\nu)=\max(I\backslash a(\nu))-1$. Then we have
\[
\sum_{i\notin a(\nu)}t_i\sim t_{j(\nu)+1}
  =\prod^{j(\nu)}_{j=0}(t_{j+1}/t_j).
\]
Hence we have
\begin{equation}
\begin{split}
\prod_\nu\biggl(\sum_{i\notin
  a(\nu)}t_i\biggr)^{\abs{a(\nu-1)}-\abs{a(\nu)}}
&\sim\prod_\nu\prod^{j(\nu)}_{j=0}
  (t_{j+1}/t_j)^{\abs{a(\nu-1)}-\abs{a(\nu)}}\\
&=\prod_{j\ge 0}(t_{j+1}/t_j)^{
  \sum_{j(\nu)\ge j}(\abs{a(\nu-1)}-\abs{a(\nu)})}.
\end{split}
\end{equation}
By definition, we have $a(\nu(j))\supset c(j)$. Hence, $\abs{c(j)}=n-j$
implies (5.4). If $c(j)\notin a$, $a(\nu(j))c(j)$ and hence
we have (5.5).
\end{proof}

\begin{quotation}
This is an example of an `extract'. The magnetization $M_0$ of the Ising
model is related to the local state probability $P(a):M_0=P(1)-P(-1)$.
The equivalences are shown in Table~\ref{eqtable}.
\end{quotation}

\begin{table}[ht]
\caption{}\label{eqtable}
\renewcommand\arraystretch{1.5}
\noindent\[
\begin{array}{|c|c|c|}
\hline
&{-\infty}&{+\infty}\\
\hline
{f_+(x,k)}&e^{\sqrt{-1}kx}+s_{12}(k)e^{-\sqrt{-1}kx}&s_{11}(k)e^
{\sqrt{-1}kx}\\
\hline
{f_-(x,k)}&s_{22}(k)e^{-\sqrt{-1}kx}&e^{-\sqrt{-1}kx}+s_{21}(k)e^{\sqrt
{-1}kx}\\
\hline
\end{array}
\]
\end{table}

\begin{definition}
This is an example of a `definition' element.
For $f\in A(X)$, we define
\begin{equation}
\mathcal{Z} (f)=\{E\in Z[X]: \text{$f$ is $E^c$-regular}\}.
\end{equation}
\end{definition}

\begin{remark}
This is an example of a `remark' element.
For $f\in A(X)$, we define
\begin{equation}
\mathcal{Z} (f)=\{E\in Z[X]: \text{$f$ is $E^c$-regular}\}.
\end{equation}
\end{remark}

\begin{example}
This is an example of an `example' element.
For $f\in A(X)$, we define
\begin{equation}
\mathcal{Z} (f)=\{E\in Z[X]: \text{$f$ is $E^c$-regular}\}.
\end{equation}
\end{example}

\begin{xca}
This is an example of the \texttt{xca} environment. This environment is
used for exercises which occur within a section.
\end{xca}

Some extra text before the \texttt{xcb} head. The \texttt{xcb} environment
is used for exercises that occur at the end of a chapter.  Here it contains
an example of a numbered list.

\begin{xcb}{Exercises}
\begin{enumerate}
\item First item.
In the case where in $G$ there is a sequence of subgroups
\[
G = G_0, G_1, G_2, \dots, G_k = e
\]
such that each is an invariant subgroup of $G_i$.

\item Second item.
Its action on an arbitrary element $X = \lambda^\alpha X_\alpha$ has the
form
\begin{equation}\label{eq:action}
[e^\alpha X_\alpha, X] = e^\alpha \lambda^\beta
[X_\alpha X_\beta] = e^\alpha c^\gamma_{\alpha \beta}
 \lambda^\beta X_\gamma,
\end{equation}

\begin{enumerate}
\item First subitem.
\[
- 2\psi_2(e) =  c_{\alpha \gamma}^\delta c_{\beta \delta}^\gamma
e^\alpha e^\beta.
\]

\item Second subitem.
\begin{enumerate}
\item First subsubitem.
In the case where in $G$ there is a sequence of subgroups
\[
G = G_0, G_1, G_2, \ldots, G_k = e
\]
such that each subgroup $G_{i+1}$ is an invariant subgroup of $G_i$ and
each quotient group $G_{i+1}/G_{i}$ is abelian, the group $G$ is called
\textit{solvable}.

\item Second subsubitem.
\end{enumerate}
\item Third subitem.
\end{enumerate}
\item Third item.
\end{enumerate}
\end{xcb}

Here is an example of a cite. See \cite{A}.

\begin{theorem}
This is an example of a theorem.
\end{theorem}

\begin{theorem}[Marcus Theorem]
This is an example of a theorem with a parenthetical note in the
heading.
\end{theorem}

\begin{figure}[tb]
\blankbox{.6\columnwidth}{5pc}
\caption{This is an example of a figure caption with text.}
\label{firstfig}
\end{figure}

\begin{figure}[tb]
\blankbox{.75\columnwidth}{3pc}
\caption{}\label{otherfig}
\end{figure}

\section{Some more list types}
This is an example of a bulleted list.

\begin{itemize}
\item $\mathcal{J}_g$ of dimension $3g-3$;
\item $\mathcal{E}^2_g=\{$Pryms of double covers of $C=\openbox$ with
normalization of $C$ hyperelliptic of genus $g-1\}$ of dimension $2g$;
\item $\mathcal{E}^2_{1,g-1}=\{$Pryms of double covers of
$C=\openbox^H_{P^1}$ with $H$ hyperelliptic of genus $g-2\}$ of
dimension $2g-1$;
\item $\mathcal{P}^2_{t,g-t}$ for $2\le t\le g/2=\{$Pryms of double
covers of $C=\openbox^{C'}_{C''}$ with $g(C')=t-1$ and $g(C'')=g-t-1\}$
of dimension $3g-4$.
\end{itemize}

This is an example of a `description' list.

\begin{description}
\item[Zero case] $\rho(\Phi) = \{0\}$.

\item[Rational case] $\rho(\Phi) \ne \{0\}$ and $\rho(\Phi)$ is
contained in a line through $0$ with rational slope.

\item[Irrational case] $\rho(\Phi) \ne \{0\}$ and $\rho(\Phi)$ is
contained in a line through $0$ with irrational slope.
\end{description}

\endinput

%-----------------------------------------------------------------------
% End of chap1.tex
%-----------------------------------------------------------------------


\backmatter
%-----------------------------------------------------------------------------
% Beginning of biblio.tex
%-----------------------------------------------------------------------------

\bibliographystyle{amsalpha}
\begin{thebibliography}{A}

\bibitem [A]{A} T. Aoki, \textit{Calcul exponentiel des op\'erateurs
microdifferentiels d'ordre infini.} I, Ann. Inst. Fourier (Grenoble)
\textbf{33} (1983), 227--250.

\bibitem [B]{B} R. Brown, \textit{On a conjecture of Dirichlet},
Amer. Math. Soc., Providence, RI, 1993.

\bibitem [D]{D} R. A. DeVore, \textit{Approximation of functions},
Proc. Sympos. Appl. Math., vol. 36,
Amer. Math. Soc., Providence, RI, 1986, pp. 34--56.

\end{thebibliography}

%-----------------------------------------------------------------------------
% End of biblio.tex
%-----------------------------------------------------------------------------

%-----------------------------------------------------------------------
% Beginning of index.tex
%-----------------------------------------------------------------------
%
%  AMS-LaTeX sample file for a monograph index, as used by AMS document
%  classes for book series.  This is a data file input by chapter.tex.
%  It was copied from a .ind file prepared for another purpose, and is
%  given here only for the purpose of showing the output format of an
%  index in this series.
%
% *** DO NOT CREATE YOUR INDEX FILE BY HAND, or use this as a model. ***
%  Put \index{...} entries in your source files.  Put \makeindex and
%  \printindex in your driver file, as shown in the template, to produce
%  a .idx file.  When the .idx file is processed by the makeindex
%  program, it will produce the .ind file with correct page numbers.
%
%%%%%%%%%%%%%%%%%%%%%%%%%%%%%%%%%%%%%%%%%%%%%%%%%%%%%%%%%%%%%%%%%%%%%%%%

\begin{theindex}

\item Absorbing barrier, 4
\item Adjoint partial differential operator, 20
\item $A$-harmonic function, 16, 182
\item $A^*$-harmonic function, 182

\indexspace

\item Boundary condition, 20, 22
\subitem Dirichlet, 15
\subitem Neumann, 16
\item Boundary value problem
\subitem the first, 16
\subitem the second, 16
\subitem the third, 16
\item Bounded set, 19

\indexspace

\item Diffusion
\subitem coefficient, 1 
\subitem equation, 3, 23
\item Dirichlet
\subitem boundary condition, 15
\subitem boundary value problem, 16

\indexspace

\item Elliptic
\subitem boundary value problem, 14, 158
\subitem partial differential equation, 14
\subitem partial differential operator, 19

\indexspace

\item Fick's law, 1
\item Flux, 1
\item Formally adjoint partial differential operator, 20
\item Fundamental solution
\subitem conceptional explanation, 12
\subitem general definition, 23
\subitem temporally homogeneous case, 64, 112

\indexspace

\item Genuine solution, 196
\item Green function, 156
\item Green's formula, 21

\indexspace

\item Harnack theorems
\subitem first theorem, 185
\subitem inequality, 186
\subitem lemma, 186
\subitem second theorem, 187
\subitem third theorem, 187
\item Helmhotz decomposition, 214
\item Hilbert-Schmidt expansion theorem, 120

\indexspace

\item Initial-boundary value problem, 22
\item Initial condition, 22
\item Invariant measure (for the fundamental solution), 167

\indexspace

\item Maximum principle
\subitem for $A$-harmonic functions, 183
\subitem for parabolic differential equations, 65
\subitem strong, 83

\indexspace

\item Neumann
\subitem boundary condition, 16
\subitem boundary value problem, 16
\subitem function, 179

\indexspace

\item One-parameter semigroup, 113

\indexspace

\item Parabolic initial-boundary value problem, 22
\item Partial differential equation
\subitem of elliptic type, 14
\subitem of parabolic type, 22
\item Positive definite kernel, 121

\indexspace

\item Reflecting barrier, 4
\item Regular (set), 19
\item Removable isolated singularity, 191
\item Robin problem, 16

\indexspace

\item Semigroup property (of fundamental solution), 64, 113
\item Separation of variables, 131
\item Solenoidal (vector field), 209
\item Strong maximum principle, 83
\item Symmetry (of fundamental solution), 64, 112

\indexspace

\item Temporally homogeneous, 111

\indexspace

\item Vector field with potential, 209

\indexspace

\item Weak solution
\subitem of elliptic equations, 195
\subitem of parabolic equation, 196
\subitem associated with a boundary condition, 204

\end{theindex}

%-----------------------------------------------------------------------
% End of index.tex
%-----------------------------------------------------------------------

\end{document}

%-----------------------------------------------------------------------
% End of chapter.tex
%-----------------------------------------------------------------------


\appendix
%    Include appendix "chapters" here.
%\include{}

\backmatter
%    Bibliographies can be prepared with BibTeX using amsplain,
%    amsalpha, or (for "historical" overviews) natbib style.
\bibliographystyle{amsplain}
%\bibliography{}

%    See note above about multiple indexes.
\printindex

\end{document}

%%%%%%%%%%%%%%%%%%%%%%%%%%%%%%%%%%%%%%%%%%%%%%%%%%%%%%%%%%%%%%%%%%%%%%%%

%    Templates for common elements of a monograph; for additional
%    information, see the AMS Author Handbook, included in every
%    AMS author package, and the amsthm user's guide, linked from
%    http://www.ams.org/tex/amslatex.html .

%    Note that [optional] short forms may be needed for running heads.

%    Chapter titles
\chapter[short form]{full title}

%    Section headings
\section[short form]{full heading}
%\subsection{}

%    Ordinary theorem and proof
\begin{theorem}[Optional addition to theorem head]
% text of theorem
\end{theorem}

\begin{proof}[Optional replacement proof heading]
% text of proof
\end{proof}

%    Figure insertion; default placement is top; if the figure occupies
%    more than 75% of a page, the [p] option should be specified.
\begin{figure}
\includegraphics{filename}
\caption{text of caption}
\label{}
\end{figure}

%    Mathematical displays; for additional information, see the amsmath
%    user's guide, linked from http://www.ams.org/tex/amslatex.html .

% Numbered equation
\begin{equation}
\end{equation}

% Unnumbered equation
\begin{equation*}
\end{equation*}

% Aligned equations
\begin{align}
  &  \\
  &
\end{align}

%-----------------------------------------------------------------------
% End of chapter-template.tex
%-----------------------------------------------------------------------
